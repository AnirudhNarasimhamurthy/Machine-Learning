\section{The EM algorithm}

The two local newspapers The Times and The Gazette publish $n$
articles everyday. The article length in the newspapers is distributed
based on the Exponential Distribution with parameter $\lambda$. That
is, for an non-negative integer x:

\[ P(wordcount = x| \lambda)=\lambda e^{-\lambda x}. \]

with parameters $\lambda_T, \lambda_G$ for the Times and the Gazette,
respectively.

({\bf Note}: Technically, using the exponential distribution is not
correct here because the exponential distribution applies to real
valued random variables, whereas here, the word counts can only be
integers. However, for simplicity, we will use the exponential
distribution instead of, say a Poisson.)

\begin{enumerate}
\item[(a)] Given an issue of one of the newspapers $(x_1,\ldots
   x_n)$, where $x_i$ denotes the length of the $i$th
   article, what is the most likely value of $\lambda$?


 \item[(b)] Assume now that you are given a collection of $m$ issues
   $\{(x_1,\ldots x_n)\}_1^m$ but you do not know which issue is a
   Times and which is a Gazette issue. Assume that the probability of
   a document is generated from the Times is $\eta$. In other words,
   it means that the probability that a document is generated from the
   Gazette is $1-\eta$.


   Explain the generative model that governs the generation of this
   data collection. In doing so, name the parameters that are required
   in order to fully specify the model.


 \item[(c)] Assume that you are given the parameters of the model
   described above. How would you use it to cluster issues to two
   groups, the Times issues and the Gazette issues?

 \item[(d)] Given the collection of $m$ issues without labels of which
   newspaper they came from, derive the update rule of the EM
   algorithm. Show all of your work.

\end{enumerate}

%%% Local Variables: 
%%% mode: latex
%%% TeX-master: "hw6"
%%% End: 
