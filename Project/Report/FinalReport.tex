% TEMPLATE for Usenix papers, specifically to meet requirements of
%  USENIX '05
% originally a template for producing IEEE-format articles using LaTeX.
%   written by Matthew Ward, CS Department, Worcester Polytechnic Institute.
% adapted by David Beazley for his excellent SWIG paper in Proceedings,
%   Tcl 96
% turned into a smartass generic template by De Clarke, with thanks to
%   both the above pioneers
% use at your own risk.  Complaints to /dev/null.
% make it two column with no page numbering, default is 10 point

% Munged by Fred Douglis <douglis@research.att.com> 10/97 to separate
% the .sty file from the LaTeX source template, so that people can
% more easily include the .sty file into an existing document.  Also
% changed to more closely follow the style guidelines as represented
% by the Word sample file. 

% Note that since 2010, USENIX does not require endnotes. If you want
% foot of page notes, don't include the endnotes package in the 
% usepackage command, below.

% This version uses the latex2e styles, not the very ancient 2.09 stuff.
\documentclass[letterpaper,twocolumn,11pt]{article}
\usepackage{usenix,epsfig,endnotes}
\usepackage{hyperref}
\usepackage{framed}
\usepackage{caption}
\usepackage{float}
\usepackage{amsmath}

\begin{document}

%don't want date printed
\date{\today}

%make title bold and 14 pt font (Latex default is non-bold, 16 pt)
\title{\Large \bf  Predict Closed Questions on Stack Overflow}

%for single author (just remove % characters)
\author{
{\rm Anirudh Narasimhamurthy}\\
University Of Utah
\and
{\rm Aravind Senguttuvan}\\
University Of Utah
% copy the following lines to add more authors
% \and
% {\rm Name}\\
%Name Institution
} % end author

\maketitle

% Use the following at camera-ready time to suppress page numbers.
% Comment it out when you first submit the paper for review.
\thispagestyle{empty}

\subsection*{Abstract}
The paper \cite{Lezina_predictclosed} uses Random Forest (RF), Support Vector Machine (SVM) and Vowpal Wabbit (VW) techniques to determine if a question should be closed. We prove the paper \cite{Lezina_predictclosed}’s evaluation.
We propose to build a classifier by ensembles and one vs all multiclass classification.

\subsection*{Classification Keywords}
referral Invite; referral Incentive; referral system; triangle counting; jaccard Similarity; referral discounts;

\section{Introduction}
StackOverflow is a question answer service which is being used by millions of programmers to get quality answers to their programming questions. After a question is submitted in StackOverflow, the proposed algorithm classifies whether a 
question will be closed or not. Classifier will output a label based on four features namely Off 
topic (OT), not constructive (NC), not a real question (NRQ), too localized (TL) or exact duplicate.
In short, the problem belongs to multiclass classification problem.



\begin{figure}[H]
\begin{center}
\includegraphics[width=1\linewidth]{labels.png}
\captionof{figure}{Labels generated by Stack overflow for closed questions}
\label{fig:referral}
\end{center}
\end{figure}

\section{Definitions}
\begin{itemize}
\item \textbf{First invites:} Exclusive invites which were initially sent out by the business. 
\item \textbf{Major players:} The users who received the first invites from the business in question certainly people who would can spread referral invites to more interested people quickest.
\end{itemize}
 
\section{Scope}
\begin{enumerate}
\item We concentrate on picking the \emph{major players} within a connected graph, assuming we are able to separate the totally disjoint clusters by centrality or clustering mechanism. (ie) There are no completely disjoint clusters in the dataset. 
\item We also assume that our problem is not to find the optimal number of invites to be sent out to a community. 
\item Our approach doesnt offer solution to find the probabilty with which a person in the community would spread their referral invites to connections so that the invites reach almost all nodes in the community.
\item Our solution deals only with first invites sent out by the business in question. We perform evaluation under this assumption.
\end{enumerate}

\section{Our Contributions}
\subsection{Presenting First Invites System}


\section{Assumptions}
\label{sec:assumptions}
\begin{enumerate}
\item The dataset has a connected graph. i.e. there is no completely disjoint clusters.
\end{enumerate}

\section{Dataset and attributes used}
\subsection{Major Dataset used for results}
Yelp Dataset: \href{http://www.yelp.com/dataset_challenge/}{link}


\subsection{Evaluation of Yelp dataset}


\begin{table*}[ht] 
\centering
\begin{tabular}{ |p{5cm}|p{3cm} |}
\hline
User names & Clustering coefficient\\ \hline
kelvin		&	1.0\\  \hline
Aaron		&	1.0\\  \hline
elizabeth	&	1.0\\  \hline
Bella		&	1.0\\  \hline
Hapi		&	1.0\\  \hline
Ellen		&	1.0\\  \hline
Chante 		&	1.0\\  \hline
Genny		&	1.0\\  \hline
Tom			&	1.0\\  \hline
David		&	1.0\\  \hline
Irene		&	1.0\\   \hline
\end{tabular}
\caption{Users with top 10 Clustering coefficient} 
\label{tab:cctop10} 
\end{table*} 




\subsubsection{Inference about the result}
From the yelp dataset, we found that although high degree approach leads to higher expected no of users early but eventually triangle counting approach led to more even distribtuion and therefore a higher final expectation value.


\subsubsection{Evaluating our method with other existing methods}


\begin{table*}[ht] 
\centering
\begin{tabular}{ |p{12cm}|p{3cm} |}
\hline 
  Major player picking Techniques & Expected number of users who have invites after 8 hops\\
  \hline 
  Triangle Counting and cluster coefficient based & 75026\\
  \hline
  Highest degree based & 73222\\
  \hline
  Random picking & 34000 \\
  \hline
\end{tabular}
\caption{Evaluation of Technique} 
\label{tab:techeval} 
\end{table*}

\section{Related work}
Some of the closely related work are noted below.

\subsection{Predict Closed Questions on StackOverflow\cite{Lezina_predictclosed}}


\subsection{Fit or Unfit : Analysis and Prediction of Closed Questions\cite{DBLP:journals/corr/CorreaS13}}


\section{Obstacles and Future Work}
\subsection{Implementation Goals}
\begin{itemize}
\item TODO
\end{itemize}

\section{Discussion of merits}

\subsection{TODO}



\section{Conclusion}


{
\small 
\bibliographystyle{acm}
\bibliography{report}
}

\end{document}